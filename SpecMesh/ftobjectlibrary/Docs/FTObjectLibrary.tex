\documentclass[9pt]{article}
\usepackage{geometry}                % See geometry.pdf to learn the layout options. There are lots.
%\usepackage{url}
\geometry{letterpaper}                   % ... or a4paper or a5paper or ... 
%\geometry{landscape}                % Activate for for rotated page geometry
%\usepackage[parfill]{parskip}    % Activate to begin paragraphs with an empty line rather than an indent
\usepackage{graphicx}
\usepackage{amssymb}
\usepackage{epstopdf}
\usepackage{listings}
\usepackage[colorlinks]{hyperref}
\usepackage{color}
%\hypersetup{citecolor=DeepPink4}
%\hypersetup{linkcolor=DarkRed}
%\hypersetup{urlcolor=blueColor}
%\usepackage{cleveref}


\textwidth = 6.5 in
\textheight = 9 in
\oddsidemargin = 0.0 in
\evensidemargin = 0.0 in
\topmargin = 0.0 in
\headheight = 0.0 in
\headsep = 0.0 in
\parskip = 0.2in
\parindent = 0.0in
\renewcommand{\familydefault}{\sfdefault}

\DeclareGraphicsRule{.tif}{png}{.png}{`convert #1 `dirname #1`/`basename #1 .tif`.png}

\title{FTObjectLibrary}
\author{David A. Kopriva}
\date{}                                           % Activate to display a given date or no date

\begin{document}
\maketitle
%\lstset{language=Fortran}          % Set your language (you can change the language for each code-block optionally)
\tableofcontents{}
\section{Introduction}
  FTObjectLibrary provides a collection of reference counted Fortran 2003 classes to 
 facilitate writing generic object oriented Fortran programs. Reference counting
 is implemented to assist with memory management so that the lifespans of objects
 are properly maintained and are so that objects are deleted only when no other references are made to them.


 FTObjectLibrary tries, as much as the maturity of Fortran compilers allow, to
 use the new F2003/2008 features to make generic programming possible. The LCD
 for the library is gfortran, and as modern features get implemented in the
 compiler, FTObjectLibrary will be updated to include those features. In the meantime, there
 are a few workarounds that exist in the code.

% The library is inspired, for the large part, by 
% \href{https://developer.apple.com/library/mac/documentation/General/Conceptual/CocoaEncyclopedia/Introduction/Introduction.html#//apple_ref/doc/uid/TP40010810-CH1-SW1}{Apple's Cocoa Foundation Framework}. 
% Reading about the philosophy behind that
% would go a long way towards understanding the why's and wherefore's of the FTObjectLibrary.

 The library includes
 three categories of classes:
 \begin{itemize}
\item Value classes
\item Container classes
\item Error reporting and testing classes
\end{itemize}
% \subsection{Value Classes}

 Value classes include the base class, FTObject and at the current time, a subclass, FTValue.
 \begin{itemize}
 \item {\bf FTObject}
 
  FTObject is the base class that implements
 the reference counting mechanism and other functions that should be overridden in subclasses.
 It the base class for all classes in the FTObjectLibrary library. You will usually not allocate objects
 of this class. Instead you will create your own subclasses of it that have data and procedures as needed.
 \item {\bf FTValue}
 
  FTValue is a wrapper class that
 allows storage of real, integer, character and logical
 values, which can then be stored in containers.
 
% \item {\bf FTData}
% 
% Is a class that allows one to store any kind of data structure.
 \end{itemize}
 
  You can create your own value
 classes by extending FTObject and store instances of those classes in the containers.

% \subsection{Container Classes}
 Container classes let you store any subclass of the base class FTObject in them. This makes it easy to
 store, for instance, a linked list of linked lists, or an array of dictionaries.
 Included in the library are the following standard container classes:
 \begin{itemize}
\item {\bf FTLinkedList}

FTLinkedList is an implementation of a doubly (and, optionally, circular) linked list.

\item {\bf FTStack}

FTStack is a subclass of FTLinkedList that adds the usual push, pop, and peek routines.

\item {\bf FTSparseMatrix}

FTSparseMatrix associates a double index (i,j) to an FTObject. Basically this is a two dimensional sparse matrix of pointers to FTObjects.
\item {\bf FTMultiIndexTable}

FTMultiIndexTable associates an integer array keys(:) to an FTObject. Basically this is an m--dimensional sparse matrix of pointers to FTObjects.

\item {\bf FTDictionary}

FTDictionary is an ``associative container'', that associates a string to another FTObject. 

\item {\bf FTValueDictionary}

FTValueDictionary is a subclass of FTDictionary that has additional methods to store and retrieve
values.

\item {\bf FTMutableObjectArray}

A mutable one-dimensional array class that can store any FTObject.

\end{itemize}

% \subsection {Error and Testing Classes}
 The library also contains classes for testing (FTAssertions, TestSuiteManagerClass) and for reporting
 errors through the FTException class.

% \subsection {Example Usage}
To give an idea about how to use the library, we show how to use a dictionary to store a real value and a string that describes the value.
In the snippet of code below, we prepare the dictionary for objects to be added to it by the init() method. We then create two values that are
added to the dictionary. 

{\color{blue}\begin{verbatim}
         SUBROUTINE constructDictionary(dict)
                 CLASS(FTDictionary), POINTER :: dict
                 CLASS(FTObject)    , POINTER :: obj
                 CLASS(FTValue)     , POINTER :: v
                 
                 ALLOCATE(dict)
                 CALL dict % initWithSize(64)
	         
                 ALLOCATE(v)
                 CALL v % initWithValue(3.14159)
                 obj => v
                 CALL dict % addObjectForKey(obj,``Pi'')
                 CALL v % release()
	         
                 ALLOCATE(v)
                 CALL v % initWithValue(``Ratio of circumference to diameter'')
                 obj => v
                 CALL dict % addObjectForKey(obj,"definition")
                 CALL v % release()
         END SUBROUTINE constructDictionary
\end{verbatim}}
 
 Notice that in the subroutine we have allocated memory for the dictionary ``dict'' and two FTValue objects. The dictionary
 is returned to the calling procedure and can be accessed from that point through the actual argument. The two value objects would normally
 be expected to be deallocated when leaving the subroutine, since otherwise they would not be accessible outside
 of the procedure. It would be dangerous if we deallocated the two objects that we created and then try to use them later through
the dictionary. This is where a systematic approach to memory management comes in. When we initialize an object, we assume ownership of it.
When we add it to the dictionary, it assumes partial ownership. So instead of deallocating the two value objects, we relinquish ownership
by way of the release() procedure, leaving only the dictionary to be responsible for deallocating them when it does not need them any more.

 We use the dictionary as shown in the next snippet of code:

{\color{blue}\begin{verbatim}
            CLASS(FTDICTIONARY), POINTER :: dict
            CLASS(FTValue)     , POINTER :: v
            REAL                         :: pi
            CHARACTER                    :: s(FTDICT_KWD_STRING_LENGTH)
            CALL constructDictionary(dict)
            
            v   => valueFromObject(dict % objectForKey("Pi"))
            pi  = v % realValue()
            
            v   => valueFromObject(dict % objectForKey(``definition''))
            s   = v % stringValue(FTDICT_KWD_STRING_LENGTH)
            PRINT *, "The num pi = ", pi," is defined as", TRIM(s)
            CALL dict % release()
            IF( dict % isUnreferenced() ) DEALLOCATE(dict)
\end{verbatim}}
  
 In this snippet, the values for the two keys ``Pi'' and ``definition'' are retrieved and then used.
 (Notice the fixed length string workaround.)
 The function valueFromObject() converts the generic object to the specific FTValue type. (If we try to convert something that is not an FTValue, the method returns an ASSOCIATED(v) == .FALSE. pointer.) Finally, the dictionary is
 released, which will release all of its objects and then deallocate all of those objects that it is
 sole owner of, which in this example is both the value and definition of $\pi$, and then DEALLOCATE it
 if it is no longer owned (referenced) by another object, which is the case here.
 
 In this way, any object that inherits from FTObject can be stored in a container, including other containers, making these
 containers quite generic, up to the limits of the Fortran language.

 \section {Memory Management}
 FTObjectLibrary uses a manual retain-release memory management model known as reference counting. A description of 
 this model can be found at 
 \url{https:developer.apple.com/library/mac/#documentation/cocoa/Conceptual/MemoryMgmt/Articles/mmRules.html#apple_ref/doc/uid/20000994-BAJHFBGH}.
 The basic idea is that a pointer object exists as long as someone ``owns'' it. When no one owns it any more, the last owner
 is responsible for deallocating the object. This approach makes it easy to keep track of when to deallocate
 pointers, and ensures that a pointer that is in use is not prematurely deallocated. In long, complex, programs, this mechanism automatically keeps track of which
 objects (pointers) that have been allocated need to be deallocated, and when.
 
 (It would be really nice if an object could detect when it should be deallocated, and deallocate itself. This does not seem to be doable in Fortran as a type bound procedure. This is the reason for the rule of deallocation by the last owner.)
  
 Ownership rules are as follows:
 \begin{itemize}
\item If you initialize an object, you own it.
\item You own an object if you call the retain() subroutine on that object.
\item You may inherit the ownership of an object by calling a method that creates (allocates and initializes) it.
\item When you no longer need an object (or are going out of scope) you must release it using the release() subroutine, and if you are the last owner, you must deallocate it.
\item You must neither relinquish ownership, nor deallocate, a pointer object that you do not own.
\item You generally do not own objects returned as pointers by functions or subroutines.
  \end{itemize}
  
\emph{Fundamentally, if you call an init() method on an object, you should call a release() and check for deallocation in the same scope.}
  
 In the example below, the main program creates a linked list to store points in, a ``point'' object (e.g. that stores x,y,z values), and adds
 the point object to the linked list. 

{\color{blue}\begin{verbatim}
      PROGRAM main
         CLASS(FTLinkedList), POINTER :: list ! Subclass of FTObject
         CLASS(Point)       , POINTER :: pnt  ! Subclass of FTObject
         CLASS(FTObject)    , POINTER :: obj
         
         ALLOCATE(list)
         CALL list % init() ! main now owns the list
         
         ALLOCATE(pnt)
         CALL pnt % initWithXYZ(0.0,0.0,0.0) ! main now owns pnt
         
         obj => pnt
         CALL list % add(obj) !list also owns pnt
         CALL pnt % release() ! main gives up ownership to pnt
         ! For the case where we are the last owner. Is .FALSE. here.
         IF ( pnt % isUnreferenced() )     THEN
            DEALLOCATE(pnt) 
         END IF 
         .
         .
         .
         ! we're done with the list, it will deallocate pnt since the list is the last owner.
         CALL list % release() 
         IF( list % isUnreferenced() )     THEN ! main IS the last owner, and must deallocate.
            DEALLOCATE(list)
         END IF

      END PROGRAM main
\end{verbatim}}

\section{Class Descriptions}

\subsection{FTObject}

 FTObject defines the basic methods that are essential for reference counted objects.
FTObject is generally not going to be instantiated by itself, but rather it will 
 be subclassed and you will work with instances of the subclasses. FTValue, for instance,
 is a subclass of FTObject.
 Otherwise, pointers of type FTObject that point to instances of subclasses
 will be stored in the container classes.


 \subsubsection{Tasks}
\begin{itemize}
\item {\bf init()}

      Initializes an object and any memory that it needs to allocate, etc. 
      Should be implemented in subclasses.The base class implementation does nothing but
      increase the reference count of the object.

\item {\bf destruct()}

      Destructor of the object, which releases and deallocates owned objects and memory.
      Should be implemented in subclasses. The base class implementation does nothing but
      decrease the reference count of the object. \emph{One does not call the destruct 
      subroutine except to call the superclass destruct in a subclasses destructor.}

\item {\bf printDescription(iUnit)}

      Prints a description of the object to a specified file unit. The base class implementation
      does nothing but print ``FTObject''

\item{\bf  copy()}

      Creates a copy (pointer) to the object of CLASS(FTObject) sourced with the object.

\item {\bf retain()}

      Increases the reference count of the object. Any procedure or object that retain()'s
      an object gains an ownership stake in that object. \emph{This procedure is not overridable.}

\item {\bf release()}

      Decreases the reference count of an object. To be called only by objects or procedures
      that have ownership in an object, i.e., for which init() or retain() have been called.
      The last owner of an object is responsible for deallocating a pointer to an object.
      \emph{This procedure is not overridable.}

\item {\bf isUnreferenced()}

      Test to see if there are no more owners of an object. If true, then the last owner is responsible
      for deallocating the pointer. \emph{This procedure is not overridable.}

\item {\bf refCount()}

      Returns the number of owners of an object. Usually this is of interest only for debugging purposes.
      \emph{This procedure is not overridable.}
\end{itemize}

 \subsubsection{Subclassing FTObject}

 In general, subclasses of FTObject implement
\begin{itemize}
\item {\bf init()}
\item {\bf destruct()}
\item {\bf printDescription()}
\end{itemize}
 
% A subclass should also provide function to convert from the base class to a subclass.
% The procedure can look something like
%
%{\color{blue}\begin{verbatim}
%      FUNCTION subclassFromObject(obj) RESULT(cast)
%         IMPLICIT NONE  
%         CLASS(FTObject), POINTER :: obj
%         CLASS(SubClass), POINTER :: cast
%         
%         cast => NULL()
%         SELECT TYPE (e => obj)
%            TYPE is (SubClass)
%               cast => e
%            CLASS DEFAULT
%               
%         END SELECT
%         
%      END FUNCTION subclassFromObject
%\end{verbatim}}


{\bf Implementing init() }

 The init() procedure performs subclass specific operations to initialize an object.

 Subclasses that override init() \emph{must} include 
 a call to the super class method. For example, if ``Subclass'' EXTENDS(FTObject), overriding init() looks like

{\color{blue}\begin{verbatim}
      SUBROUTINE initSubclass(self) 
         IMPLICIT NONE
         CLASS(Subclass) :: self
         
         CALL self % FTObject % init()
         Allocate and initialize all member objects
         ... Other Subclass specific code
      END SUBROUTINE initSubclass
\end{verbatim}}

You can have multiple initializers for an object, so it is also worthwhile understanding the concept of the ``designated initializer''. Generally speaking this is the
initializer that has the most arguments. It is also the only one that includes the call to the super class init procedure. For example, the designated initializer for a ``point'' class would be the one that takes the (x,y,z) values.

{\color{blue}\begin{verbatim}
      SUBROUTINE initPointWithXYZ(self,x,y,z) 
         IMPLICIT NONE
         CLASS(Subclass) :: self
         
         CALL self % FTObject % init()
         self % x = x
         self % y = y
         self % z = z
      END SUBROUTINE initPointWithXYZ
\end{verbatim}}

Other initializers might be the default, and the initializer that takes an array of length three. They will
do nothing but call the designated initializer. The default initializer sets the location to the origin, or some other
reasonable value.

{\color{blue}\begin{verbatim}
      SUBROUTINE initPoint(self) 
         IMPLICIT NONE
         CLASS(Subclass) :: self
         
         call self % initPointWithXYZ(0.0,0.0,0.0)
      END SUBROUTINE initPointWithXYZ
\end{verbatim}}

The array initializer is
{\color{blue}\begin{verbatim}
      SUBROUTINE initPointWithArray(self,w) 
         IMPLICIT NONE
         CLASS(Subclass) :: self
         REAL            :: w(3)
         
         call self % initPointWithXYZ(w(1),w(2),w(3))
      END SUBROUTINE initPointWithArray
\end{verbatim}}

{\bf Implementing destruct()}

 The destruct() procedure reverses the operations done in the init() procedure. It releases and
 deallocates any pointers that it owns.  Subclasses that override destruct() \emph{must} include 
 a call to the superclass method. For example, if ``Subclass'' EXTENDS(FTObject) then overriding destruct looks like

{\color{blue}\begin{verbatim}
      SUBROUTINE destructSubclass(self) 
         IMPLICIT NONE
         CLASS(Subclass) :: self
         
         Release and deallocate (if necessary) all member objects
         CALL self % FTObject % destruct()
         
      END SUBROUTINE destructSubclass
\end{verbatim}}

{\bf Implementing printDescription(iUnit)}

 printDescription is a method whose existence is to support debugging. Call printDescription(iUnit)
 on any objects owned by self for a cascading of what is stored in the object.


{\bf Converting an object from the base to a subclass}

 Container classes and the copy function return pointers to a CLASS(FTObject). To use
 any subclass features one must ``cast'' or convert a pointer to a pointer to the subclass. We like to have a specific 
 cast routine to do this as painlessly as possible. Each subclass should include a 
 function like this:

{\color{blue}\begin{verbatim}
      FUNCTION subclassFromSuperclass(obj) RESULT(cast)
         IMPLICIT NONE  
         CLASS(FTObject), POINTER :: obj
         CLASS(Subclass), POINTER :: cast
         cast => NULL()
         SELECT TYPE (e => obj)
            TYPE is (Subclass)
               cast => e
            CLASS DEFAULT
         END SELECT
      END FUNCTION subclassFromSuperclass
\end{verbatim}}

You saw an example above in the ``valueFromObject(obj)'' function.
\subsection{FTValue}
 FTValue is a not completely F2003/2008 version of an immutable class
 to store primitive values: integer, real, double precision, logical,
 character. (To Add: complex)

 This version does not use CLASS(*) or deferred length strings
 so that it can be used with gfortran 4.7/4.8. That's a shame, because both are really useful features to have.

 \subsubsection{Usage}
\begin{itemize}
\item Initialization

{\color{blue}\begin{verbatim}
            TYPE(FTValue) :: r, i, s, l, d

            CALL r % initWithValue(3.14)
            CALL i % initWithValue(6)
            CALL d % initWithValue(3.14d0)
            CALL l % initWithValue(.true.)
            CALL s % initWithValue("A string")
\end{verbatim}}

\item Destruction 

{\color{blue}\begin{verbatim}
            CALL r % destruct()   *See method definition
\end{verbatim}}

\item Accessors

{\color{blue}\begin{verbatim}
            real = r % realValue()
            int  = i % integerValue()
            doub = d % doublePrecisionValue()
            logc = l % logicalValue()
            str  = s % stringValue(nChars)
\end{verbatim}}

\item Description

{\color{blue}\begin{verbatim}
            str = v % description()
            call v % printDescription(unit)
\end{verbatim}}

\item Converting a base FTObject to an FTValue

{\color{blue}\begin{verbatim}
            CLASS(FTValue) , POINTER :: v
            CLASS(FTObject), POINTER :: obj
            v => valueFromObject(obj)
\end{verbatim}}
\end{itemize} 
 The class will attempt to convert between the different types:

{\color{blue}\begin{verbatim}
            CALL r % initWithReal(3.14)
            print *, r % stringValue(8)

            Logical variables rules:

            real, doublePrecision, integer values
            logicalValue = .FALSE. if input = 0
            logicalValue = .TRUE.  if input /= 0
\end{verbatim}}

 String values can be converted to numeric types. If the string is
 not a numeric, $Huge(x)$ will be returned, where $x$ is of the
 requested type.

\subsection{Linked Lists}

A linked list is a dynamic container that links the objects added to it in a chain. The chain can be of 
any length and objects can be added or removed at any time and at any location in the chain. Linked lists
are useful when you don't know how many objects will be stored. They are fast to add or delete
objects from, but they are slow to access any given object. To access objects, you start at the beginning
of the chain and then follow the chain it until you reach the desired entry. You use a linked list when sequential
access is more typical than random access. Special classes of linked lists are singly linked lists, which
can be followed in only one direction from the start, double linked lists that can be followed in either direction,
and circular lists where the ends are connected.

\subsubsection{FTLinkedList}
 FTLinkedList is a container class that stores objects in a doubly linked list. Optionally, the list
 can be made circular. As usual, FTLinkedList inherits from FTObjectClass.
 
 
      {\bf Definition (Subclass of FTObject):}
	{\color{blue}\begin{verbatim}
               TYPE(FTLinkedList) :: list
	\end{verbatim}}
	
     { \bf Usage:}
     
\begin{itemize}
         \item Initialization

		{\color{blue}\begin{verbatim}
	               CLASS(FTLinkedList), POINTER :: list
	               ALLOCATE(list)
	               CALL list % init()
		\end{verbatim}
}
         \item Adding an object

		{\color{blue}\begin{verbatim}
	               CLASS(FTLinkedList), POINTER :: list
	               CLASS(FTObject)    , POINTER :: obj
	
	               obj => r                ! r is subclass of FTObject
	               CALL list % Add(obj)    ! Pointer is retained by list
	               CALL obj % release()    ! If caller reliquishes ownership
		\end{verbatim}}
		
         \item Combining lists

		{\color{blue}\begin{verbatim}
	               CLASS(FTLinkedList), POINTER :: list, listToAdd
	               CLASS(FTObject)    , POINTER :: obj
	               .
	               .
	               .
	               CALL list % addObjectsFromList(listToAdd)
		\end{verbatim}}

         \item Inserting objects

		{\color{blue}\begin{verbatim}
	               CLASS(FTLinkedList)      , POINTER :: list
	               CLASS(FTObject)          , POINTER :: obj
	               CLASS(FTLinkedListRecord), POINTER :: record
	
	               obj => r                            ! r is subclass of FTObject
	               CALL list % insertObjectAfterRecord(obj,record) ! Pointer is retained by list
	               CALL obj % release()                ! If caller reliquishes ownership
		\end{verbatim}}
		OR
		{\color{blue}\begin{verbatim}
	               CLASS(FTLinkedList)      , POINTER :: list
	               CLASS(FTObject)          , POINTER :: obj, otherObject
	               CLASS(FTLinkedListRecord), POINTER :: record
	
	               obj => r                            ! r is subclass of FTObject
	               CALL list % insertObjectAfterObject(obj,otherObject) ! Pointer is retained by list
	               CALL obj % release()                ! If caller reliquishes ownership
		\end{verbatim}}
		

         \item Removing objects
		{\color{blue}\begin{verbatim}
	               CLASS(FTLinkedList), POINTER :: list
	               CLASS(FTObject)    , POINTER :: obj
	               obj => r                 ! r is subclass of FTObject
	               CALL list % remove(obj)
		\end{verbatim}}
		
         \item Getting an array of all of the objects

		{\color{blue}\begin{verbatim}
	               CLASS(FTLinkedList)         , POINTER :: list
               CLASS(FTMutableObjectArray) , POINTER :: array
               array => list % allObjects() ! Array has refCount = 1
		\end{verbatim}}

	\item Making a linked list circular or not. The default is not circular.
		{\color{blue}\begin{verbatim}
	              LOGICAL :: c = .true.
	              CALL list % makeCircular( c )
		\end{verbatim}}
		
	\item Checking to see if a linked list circular or not
		{\color{blue}\begin{verbatim}
	              LOGICAL :: c 
	              c = list % isCircular()
		\end{verbatim}}
		
         \item Counting the number of objects in the list

		{\color{blue}\begin{verbatim}
		               n = list % count()
		\end{verbatim}}
		
         \item Reversing the order of the list

		{\color{blue}\begin{verbatim}
		               CALL list % reverse()
		\end{verbatim}}

         \item Destruction
         
		{\color{blue}\begin{verbatim}
		               CALL list % release()
		               IF ( list % isUnreferenced() )     THEN ! If list is a pointer
		                  DEALLOCATE(list)
		                  list => NULL()
		               END IF
		\end{verbatim}}
\end{itemize}

\subsubsection{FTLinkedListIterator}
Linked lists must be navigated in order along the chain. This is usually accomplished with the help of
an \emph{iterator} to navigate linked lists. FTObjectLibrary includes a class called FTLinkedListIterator for stepping through (iterating) a linked
list to access its entries.


      {\bf Definition (Subclass of FTObject):}
	{\color{blue}\begin{verbatim}
               TYPE(FTLinkedListIterator) :: iterator
	\end{verbatim}}
	
     { \bf Usage:}
\begin{itemize}
         \item Initialization

{\color{blue}\begin{verbatim}
              CLASS(FTLinkedListIterator), POINTER :: iterator
              CLASS(FTLinkedList)        , POINTER :: list
              ALLOCATE(iterator)
              CALL iterator % init()
              .
              .
              .
              CALL iterator % setLinkedList(list) ! Increases refCount of list
\end{verbatim}}
Or
{\color{blue}\begin{verbatim}
              CLASS(FTLinkedList)        , POINTER :: list
              CLASS(FTLinkedListIterator), POINTER :: iterator
              ALLOCATE(iterator)
              .
              .
              .
              CALL iterator % initWithFTLinkedList(list) ! Increases refCount of list
\end{verbatim}}

         \item Setting the iterator to the beginning of the list

{\color{blue}\begin{verbatim}
               CALL iterator % setToStart()
\end{verbatim}}

\item Testing if the iterator is at the end of the list

{\color{blue}\begin{verbatim}
               IF( iterator % isAtEnd() )
\end{verbatim}}

\item Iterating through the list and accessing contained objects
{\color{blue}\begin{verbatim}
              CLASS(FTObject), POINTER :: obj
              CALL iterator % setToStart()
              DO WHILE (.NOT.iterator % isAtEnd())
                  obj => iterator % object()              ! if the object is wanted
                  recordPtr => iterator % currentRecord() ! if the record is wanted
                  
                  !Do something with object or record

                  CALL iterator % moveToNext() ! FORGET THIS CALL AND YOU GET AN INFINITE LOOP!
              END DO
\end{verbatim}}

         \item Destruction

{\color{blue}\begin{verbatim}
               CALL iterator % release()
               IF ( iterator % isUnreferenced() )     THEN ! if it is a pointer
                  DEALLOCATE(iterator) 
                  iterator => null()
               END IF
\end{verbatim}}
\end{itemize}

\subsection{Stacks}
A stack is a data structure that enforces last-in/first-out access to the objects stored in it. One puts
a new object onto the top of the stack with a \emph{push} operation, and pulls off the object at the
top of the stack with a \emph{pop} operation. Usually one has the option to just look at the top of 
the stack with a \emph{peek} operation that doesn't remove the top object. You would implement a 
Reverse Polish calculator with a stack, for instance.

     {\bf Definition (Subclass of FTLinkedListClass):}
	{\color{blue}\begin{verbatim}
        	TYPE(FTStack) :: stack
	\end{verbatim}}
     {\bf Usage:}

\begin{itemize}
        \item Initialization
	{\color{blue}\begin{verbatim}
           ALLOCATE(stack) ! If stack is a pointer
           CALL stack  %  init()
	\end{verbatim}}
        \item Destruction
	{\color{blue}\begin{verbatim}
           CALL stack  %  release()
           IF ( stack  %  isUnreferenced() )     THEN
              DEALLOCATE(stack) ! If stack is a pointer
              stack => NULL()
           END IF 
	\end{verbatim}}
        \item Pushing an object onto the stack

	{\color{blue}\begin{verbatim}
           TYPE(FTObject) :: obj
           obj => r1
           CALL stack % push(obj)
	\end{verbatim}}
        \item Peeking at the top of the stack
	{\color{blue}\begin{verbatim}
           obj => stack % peek() ! No change of ownership
           SELECT TYPE(obj)
              TYPE is (*SubclassType*)
                 � Do something with obj as subclass
              CLASS DEFAULT
                 � Problem with casting
           END SELECT
	\end{verbatim}}
        \item Popping the top of the stack
	{\color{blue}\begin{verbatim}
           obj => stack % pop() ! Ownership transferred to caller
	\end{verbatim}}
\end{itemize}
\subsection{Object Arrays}
Fortran has pointers to arrays, but not arrays of pointers. To do the latter, one creates
a wrapper derived type and creates an array of that wrapper type. Fortran arrays are great, but
they are of fixed length, and they don't easily implement reference counting to keep track of
memory. For that, we have the FTMutableObjectArray. Performance reasons dictate that you 
will use regular arrays for numeric types and the like, but for generic objects we would use
an Object Array.

You initialize a FTMutableObjectArray with the number of objects that you expect it to hold.
However, it can re-size itself if necessary. To be efficient, it adds more than one entry at a time
given by the ``chunkSize'', which you can choose for yourself. (The default is 10.)

     {\bf Definition (Subclass of FTObject):}
	{\color{blue}\begin{verbatim}
        	TYPE(FTMutableObjectArray) :: array
	\end{verbatim}}
     {\bf Usage:}

\begin{itemize}
        \item Initialization
		{\color{blue}\begin{verbatim}
      CLASS(FTMutableObjectArray)  :: array
      INTEGER                      :: N = 11
      CALL array % initWithSize(N)
	\end{verbatim}}
	
        \item Destruction
		{\color{blue}\begin{verbatim}
           CALL array  %  release()
           IF ( array  %  isUnreferenced() )     THEN
              DEALLOCATE(array) ! If array is a pointer
              array => NULL()
           END IF 
	\end{verbatim}}
	
        \item Adding an object

	{\color{blue}\begin{verbatim}
           TYPE(FTObject) :: obj
           obj => r1
           CALL array % addObject(obj)
	\end{verbatim}}
	
        \item Removing an object
	{\color{blue}\begin{verbatim}
           TYPE(FTObject) :: obj
           CALL array % removeObjectAtIndex(i)
	\end{verbatim}}
		
        \item Accessing an object
	{\color{blue}\begin{verbatim}
           TYPE(FTObject) :: obj
           obj => array % objectAtIndex(i)
	\end{verbatim}}

        \item Replacing an object
	{\color{blue}\begin{verbatim}
           TYPE(FTObject) :: obj
           obj => r1
           CALL array % replaceObjectAtIndexWithObject(i,obj)
	\end{verbatim}}
	
        \item Setting the chunk size
	{\color{blue}\begin{verbatim}
           CALL array % setChunkSize(size)
	\end{verbatim}}
	
        \item Getting the chunk size
	{\color{blue}\begin{verbatim}
           i = array % chunkSize(size)
	\end{verbatim}}
	
        \item Finding the number of items in the array
	{\color{blue}\begin{verbatim}
           n =  array % count()
	\end{verbatim}}
	
        \item Finding the actual allocated size of the array
	{\color{blue}\begin{verbatim}
           n =  array % allocatedSize()
	\end{verbatim}}
	
        \item Converting a base class pointer to an object array 
	{\color{blue}\begin{verbatim}
           Array =>  objectArrayFromObject(obj)
	\end{verbatim}}
	
\end{itemize}

\subsection{Sparse Matrices}
Hash tables are data structures designed to enable storage and fast
retrieval of key-value pairs. An example of a key-value pair is
a variable name (``gamma'') and its associated value (``1.4'').
The table itself is typically an array.
The location of the value in a hash table associated with
a key, $k$, is specified by way of a \emph{hash function}, $H(k)$.
In the case of a variable name and value, the hash function
would convert the name into an integer that tells where to
find the associated value in the table.

A very simple example of a
hash table is, in fact, a singly dimensioned array. The key is 
the array index and the value is what is stored at that index.
Multiple keys can be used to identify data; a two dimensional
array provides an example of where two keys are used to access memory
and retrieve the value at that location.
If we view a singly dimensioned array as a special case of a hash table,
its hash function is just the array index, $H(j)=j$. A doubly dimensioned array
could be (and often is) stored columnwise as a singly dimensioned array by creating a hash
function that maps the two indices to a single location in the array, e.g.,
$H(i,j) = i + j*N$, where $N$ is the range of the first index, $i$. 

Two classes are included in FTObjectLibrary. The first, FTSparseMatrix, works with an ordered pair, (i,j), as the
keys. The second, FTMultiIndexTable, uses an array of integers as the keys.

Both classes include enquiry functions to see of an object exists for the given keys. Otherwise,
the function that returns an object for a given key will return an UNASSOCIATED pointer if there
is no object for the key. Be sure to retain any object returned by the objectForKeys methods if 
you want to keep it beyond the lifespan of the matrix or table. For example,

	{\color{blue}\begin{verbatim}
           TYPE(FTObject) :: obj
           obj => matrix % objectForKeys(i,j)
           IF ( ASSOCIATED(OBJ) ) THEN
           	 CALL obj % retain()
	             ! Cast obj to something useful
           ELSE
             ! Perform some kind of error recovery
           END IF 
        	\end{verbatim}}



\subsubsection{FTSparseMatrix}
The sparse matrix included in the FTObjectLibrary is very simple in that it has a predefined
hash function with two keys, $(i,j)$. You will initialize the matrix
with the number of rows.

     {\bf Definition (Subclass of FTObject):}
	{\color{blue}\begin{verbatim}
        	TYPE(FTSparseMatrix) :: matrix
	\end{verbatim}}
     {\bf Usage:}

\begin{itemize}
        \item Initialization
		{\color{blue}\begin{verbatim}
      CLASS(FTSparseMatrix)  :: matrix
      INTEGER                :: N = 11 ! Number of rows
      CALL matrix % initWithSize(N)
	\end{verbatim}}
	
        \item Destruction
		{\color{blue}\begin{verbatim}
           CALL matrix  %  release()
           IF ( matrix  %  isUnreferenced() )     THEN
              DEALLOCATE(matrix) ! If matrix is a pointer
              matrix => NULL()
           END IF 
	\end{verbatim}}
	
        \item Adding an object

	{\color{blue}\begin{verbatim}
           TYPE(FTObject) :: obj
           matrix % addObjectForKeys(obj,i,j)
	\end{verbatim}}
	
        \item Accessing an object

	{\color{blue}\begin{verbatim}
           TYPE(FTObject) :: obj
           obj => matrix % objectForKeys(i,j)
	\end{verbatim}}
	
        \item Checking if an entry exists
	{\color{blue}\begin{verbatim}
           LOGICAL :: exists
           exists = matrix % containsKeys(i,j)
	\end{verbatim}}
	
\end{itemize}
\subsubsection{FTMultiIndexTable}
An extension (not in the subclass sense) of the sparse matrix class is the MultiIndexTable. It uses an integer array
of keys instead of just an (i,j) pair. A multiIndexTable can be used, for instance, to determine if the four node ids of
a face of an element match those of another face of another element.

     {\bf Definition (Subclass of FTObject):}
	{\color{blue}\begin{verbatim}
        	TYPE(FTMultiIndexTable) :: table
	\end{verbatim}}
     {\bf Usage:}

\begin{itemize}
        \item Initialization
		{\color{blue}\begin{verbatim}
      CLASS(FTMultiIndexTable)  :: table
      INTEGER                   :: N = 11 ! maximum over all keys
      CALL table % initWithSize(N)
	\end{verbatim}}
	
        \item Destruction
		{\color{blue}\begin{verbatim}
           CALL table  %  release()
           IF ( table  %  isUnreferenced() )     THEN
              DEALLOCATE(table) ! If table is a pointer
              table => NULL()
           END IF 
	\end{verbatim}}
	
        \item Adding an object

	{\color{blue}\begin{verbatim}
           TYPE(FTObject) :: obj
           INTEGER        :: keys(m) ! m = # keys
           matrix % addObjectForKeys(obj,keys)
	\end{verbatim}}
	
        \item Accessing an object

	{\color{blue}\begin{verbatim}
           TYPE(FTObject) :: obj
           INTEGER        :: keys(m) ! m = # keys
           obj => matrix % objectForKeys(keys)
	\end{verbatim}}
	
        \item Checking if an entry exists
	{\color{blue}\begin{verbatim}
           LOGICAL :: exists
           INTEGER :: keys(m) ! m = # keys
           exists = matrix % containsKeys(keys)
	\end{verbatim}}
	
\end{itemize}

\subsection{Dictionaries}
A dictionary is a special case of a hash table that stores \emph{key-value pairs}. It is an
example of what is called an ``associative container''. In the implementation of FTObjectLibrary,
the value can be any subclass of FTObject and the key is a character variable. The library
includes the base dictionary that can store and retrieve any subclass of FTObject. It also includes a
subclass that is designed to store and retrieve FTValue objects.

\subsubsection{FTDictionary}
     {\bf Definition (Subclass of FTObject):}
	{\color{blue}\begin{verbatim}
        	TYPE(FTDictionary) :: dict
	\end{verbatim}}
     {\bf Usage:}

\begin{itemize}
        \item Initialization
	{\color{blue}\begin{verbatim}
      CLASS(FTDictionary)  :: dict
      INTEGER              :: N = 16 ! Should be a power of two.
      CALL dict % initWithSize(N)
	\end{verbatim}}
	
        \item Destruction
		{\color{blue}\begin{verbatim}
           CALL dict  %  release()
           IF ( dict  %  isUnreferenced() )     THEN
              DEALLOCATE(dict) ! If table is a pointer
              dict => NULL()
           END IF 
	\end{verbatim}}
	
        \item Adding a key-object pair

	{\color{blue}\begin{verbatim}
              CLASS(FTDictionary), POINTER :: dict
              CLASS(FTObject)    , POINTER :: obj
              CHARACTER(LEN=M)             :: key
              obj => r                            ! r is subclass of FTObject
              CALL dict % addObjectForKey(obj,key)
	\end{verbatim}}
	
        \item Accessing an object

	{\color{blue}\begin{verbatim}
           TYPE(FTObject) :: obj
           obj => dict % objectForKey(key)
	\end{verbatim}}
	
        \item Converting a base class pointer to a dictionary
	{\color{blue}\begin{verbatim}
           dict =>  dictionaryFromObject(obj)
	\end{verbatim}}
		
        \item Getting all of the keys (The target of the pointer must be deallocated by the caller)
	{\color{blue}\begin{verbatim}
           CHARACTER(LEN=FTDICT_KWD_STRING_LENGTH), POINTER :: keys(:)
           keys =>  dict % allKeys()
	\end{verbatim}}
	
        \item Getting all of the objects
	{\color{blue}\begin{verbatim}
           CLASS(FTMutableObjectArray), POINTER :: objectArray
           objectArray =>  dict % allObjects() ! The array is owned by the caller.
	\end{verbatim}}
	
\end{itemize}
\subsubsection{FTValueDictionary}
FTValueDictionary adds methods to store and retrieve FTValue objects.

     {\bf Definition (Subclass of FTDictionary):}
	{\color{blue}\begin{verbatim}
        	TYPE(FTValueDictionary) :: dict
	\end{verbatim}}
     {\bf Usage:}
\begin{itemize}
\item Adding a value
	{\color{blue}\begin{verbatim}
    CALL dict % addValueForKey(1,"integer")
    CALL dict % addValueForKey(3.14,"real")
    CALL dict % addValueForKey(98.6d0,"double")
    CALL dict % addValueForKey(.true.,"logical")
    CALL dict % addValueForKey("Hello World","string")
	\end{verbatim}}
	
\item Accessing a value
	{\color{blue}\begin{verbatim}
     i = dict % integerValueForKey("integer")
     r = dict % realValueForKey("real")
     d = dict % doublePrecisionValueForKey("double")
     l = dict % logicalValueForKey("logical")
     s = dict % stringValueForKey("string",15)
	\end{verbatim}}
	
	Note that the FTValue class will do type conversion, so you can also access something like
	{\color{blue}\begin{verbatim}
     i = dict % integerValueForKey("real")
     s = dict % stringValueForKey(``real'',15)
	\end{verbatim}}
\item Converting an FTDictionary to an FTValueDictionary
	{\color{blue}\begin{verbatim}
        	 valueDict => valueDictionaryFromDictionary(dict)
	\end{verbatim}}
\item Converting an FTObject to an FTValueDictionary
	{\color{blue}\begin{verbatim}
        	 valueDict => valueDictionaryFromObject(obj)
	\end{verbatim}}

\end{itemize}
\section{Advanced Classes: Testing and Error Reporting}

The library includes classes for testing and reporting errors. Errors are reported through
instances of the FTException class. Testing is made semi-automatic through the
TestSuiteManager class and procedures defined in the FTAssertions module.
\subsection{Assertions}
An assertion is a true-false statement that you expect to be true. Assertions are used
to test for exceptional situations (AKA ``Failures'') in a code. For example, knowing that density
always must be positive, you might assert that fact before using it, and if the result is false 
generate an error. With FTObjectLibrary that would be
	{\color{blue}\begin{verbatim}
        	CALL assert(rho > 0,``Density must be positive'')
	\end{verbatim}}


Fortran does not have an assertion mechanism, and so pretty much everyone writes their own. 
There are a couple of open source projects available, but one never knows how actively they will be 
maintained. In the grand Fortran tradition of writing one's own, FTObjectLibrary has
an (incomplete) assertion module. 

To use assertions, you will USE the FTAssertions module and initialize the assertions system by calling
	{\color{blue}\begin{verbatim}
        	CALL initializeSharedAssertionsManager
	\end{verbatim}}
During the course of your program, the sharedAssertionsManager will keep track of the 
success or failure of the assertions that you make. You can enquire at any time how many assertions
have failed and how many assertions have been made with the two enquiry functions
	{\color{blue}\begin{verbatim}
     INTEGER FUNCTION numberOfAssertionFailures()
     INTEGER FUNCTION numberOfAssertions()
	\end{verbatim}}
You can get a summary of the assertions by calling the subroutine
	{\color{blue}\begin{verbatim}
     SUBROUTINE SummarizeFTAssertions(title,iUnit)  
        IMPLICIT NONE
        CHARACTER(LEN=*)                        :: title
        INTEGER                                 :: iUnit
	\end{verbatim}}
When you are done, you finalize the sharedAssertionsManager with
	{\color{blue}\begin{verbatim}
        	CALL finalizeSharedAssertionsManager
	\end{verbatim}}

So how do you make assertions? FTObjectLibrary supplies two subroutines that post 
failures to the sharedAssertionsManager. The first takes a LOGICAL variable
	{\color{blue}\begin{verbatim}
      SUBROUTINE assert(test,msg)  
         IMPLICIT NONE
         CHARACTER(LEN=*), OPTIONAL :: msg
         LOGICAL                    :: test
	\end{verbatim}}
The second tests equality through the overloaded subroutine assertEqual, which allows a variety 
of argument type listed below:
	{\color{blue}\begin{verbatim}
       INTERFACE assertEqual
         MODULE PROCEDURE assertEqualTwoIntegers
         MODULE PROCEDURE assertEqualTwoIntegerArrays1D
         MODULE PROCEDURE assertEqualTwoIntegerArrays2D
         MODULE PROCEDURE assertWithinToleranceTwoReal
         MODULE PROCEDURE assertWithinToleranceTwoRealArrays1D
         MODULE PROCEDURE assertWithinToleranceTwoRealArrays2D
         MODULE PROCEDURE assertWithinToleranceTwoDouble
         MODULE PROCEDURE assertWithinToleranceTwoDoubleArrays1D
         MODULE PROCEDURE assertWithinToleranceTwoDoubleArrays2D
         MODULE PROCEDURE assertEqualTwoLogicals
         MODULE PROCEDURE assertEqualString
      END INTERFACE assertEqual
	\end{verbatim}}
The individual calls have the signatures
	{\color{blue}\begin{verbatim}
      SUBROUTINE assertEqualTwoIntegers(expectedValue,actualValue,msg)  
         IMPLICIT NONE  
         INTEGER, INTENT(in)        :: expectedValue,actualValue
         CHARACTER(LEN=*), OPTIONAL :: msg
         
      SUBROUTINE assertEqualTwoIntegerArrays1D(expectedValue,actualValue)  
         IMPLICIT NONE  
         INTEGER, INTENT(in)    , DIMENSION(:)            :: expectedValue,actualValue
         
      SUBROUTINE assertEqualTwoIntegerArrays2D(expectedValue,actualValue)  
         IMPLICIT NONE  
         INTEGER, INTENT(in)    , DIMENSION(:,:)          :: expectedValue,actualValue
      SUBROUTINE assertWithinToleranceTwoReal(x,y,tol,msg)  
         IMPLICIT NONE  
         REAL, INTENT(in)           :: x,y,tol
         CHARACTER(LEN=*), OPTIONAL :: msg
         
      SUBROUTINE assertWithinToleranceTwoRealArrays1D(expectedValue,actualValue,tol,msg)  
         IMPLICIT NONE  
         REAL, INTENT(IN), DIMENSION(:) :: expectedValue,actualValue
         REAL, INTENT(IN)               :: tol
         CHARACTER(LEN=*), OPTIONAL     :: msg
         
      SUBROUTINE assertWithinToleranceTwoRealArrays2D(expectedValue,actualValue,tol)  
         IMPLICIT NONE  
         REAL, INTENT(IN), DIMENSION(:,:) :: expectedValue,actualValue
         REAL, INTENT(IN)                 :: tol
         
      SUBROUTINE assertWithinToleranceTwoDouble(expectedValue,actualValue,tol,msg)  
         IMPLICIT NONE  
         DOUBLE PRECISION, INTENT(in) :: expectedValue,actualValue,tol
         CHARACTER(LEN=*), OPTIONAL   :: msg
         
      SUBROUTINE assertWithinToleranceTwoDoubleArrays1D(expectedValue,actualValue,tol,msg)  
         IMPLICIT NONE  
         DOUBLE PRECISION, INTENT(IN), DIMENSION(:) :: expectedValue,actualValue
         DOUBLE PRECISION, INTENT(IN)               :: tol
         CHARACTER(LEN=*), OPTIONAL                 :: msg
         
      SUBROUTINE assertWithinToleranceTwoDoubleArrays2D(expectedValue,actualValue,tol)  
         IMPLICIT NONE  
         DOUBLE PRECISION, INTENT(IN), DIMENSION(:,:) :: expectedValue,actualValue
         DOUBLE PRECISION, INTENT(IN)                 :: tol
         
      SUBROUTINE assertEqualString(expectedValue,actualValue,msg)
         IMPLICIT NONE
         CHARACTER(LEN=*)           :: expectedValue,actualValue
         CHARACTER(LEN=*), OPTIONAL :: msg
         
      SUBROUTINE assertEqualTwoLogicals(expectedValue,actualValue,msg)  
         IMPLICIT NONE  
         LOGICAL, INTENT(in)        :: expectedValue,actualValue
         CHARACTER(LEN=*), OPTIONAL :: msg
	\end{verbatim}}
	Notice that you can only check the equality of two floating point numbers to within some tolerance.
\subsection{Testing}
FTObjectLibrary also includes a testing suite with which you can create a suite of tests
to make sure your codes are working and stay working. The tests are managed by an instance of the
{\color{blue}{TestSuiteManager}} class. It is designed to be used with minimal fuss. You
\begin{enumerate}
\item Initialize the test suite
\item Add test subroutines
\item Have the testSuiteManager perform the tests
\item Finalize the test suite manager
\end{enumerate}


An example of running a suite of tests is the following:
	{\color{blue}\begin{verbatim}
      TYPE(TestSuiteManager) :: testSuite
      
      EXTERNAL :: FTDictionaryClassTests
      EXTERNAL :: FTExceptionClassTests
      EXTERNAL :: FTValueClassTests
      EXTERNAL :: FTValueDictionaryClassTests
      EXTERNAL :: FTLinkedListClassTests
      EXTERNAL :: StackClassTests
      EXTERNAL :: MutableArrayClassTests
      EXTERNAL :: HashTableTests

      CALL testSuite % init()
      
      CALL testSuite % addTestSubroutineWithName(FTValueClassTests,"FTValueClass Tests")
      CALL testSuite % addTestSubroutineWithName(FTDictionaryClassTests,"FTDictionaryClass Tests")
      CALL testSuite % addTestSubroutineWithName(FTValueDictionaryClassTests,"FTValueDictionaryClass Tests")
      CALL testSuite % addTestSubroutineWithName(FTLinkedListClassTests,"FTLinkedListClass Tests")
      CALL testSuite % addTestSubroutineWithName(StackClassTests,"StackClass Tests")
      CALL testSuite % addTestSubroutineWithName(MutableArrayClassTests,"Mutable Array Tests")
      CALL testSuite % addTestSubroutineWithName(HashTableTests,"Hash Table Tests")
      CALL testSuite % addTestSubroutineWithName(FTExceptionClassTests,"FTExceptionClass Tests")

      CALL testSuite % performTests()

      CALL testSuite % finalize()
	\end{verbatim}}
The test subroutines have no arguments. The interface is
{\color{blue}\begin{verbatim}
      ABSTRACT INTERFACE
         SUBROUTINE testSuiteFunction()
         END SUBROUTINE testSuiteFunction
      END INTERFACE
\end{verbatim}}

The test functions should USE the FTAssertions module as in the previous section. You don't have to do any reporting code in your tests, however. Reporting is managed by the testSuiteManager at the end of performTests.

     {\bf Definition:}
	{\color{blue}\begin{verbatim}
        	TYPE(TestSuiteManager) :: tester
	\end{verbatim}}
     {\bf Usage:}
\begin{itemize}
\item Initialization
	{\color{blue}\begin{verbatim}
    CALL tester % init()
	\end{verbatim}}
	
\item Creating a test
   Create subroutines with the interface
{\color{blue}\begin{verbatim}
      ABSTRACT INTERFACE
         SUBROUTINE testSuiteFunction()
         END SUBROUTINE testSuiteFunction
      END INTERFACE
\end{verbatim}}
that (typically) includes unit test calls. 
\item Adding a test case
	{\color{blue}\begin{verbatim}
         CALL tester % addTestSubroutineWithName(SubroutineName, description)
	\end{verbatim}}
where
\begin{itemize}
\item SubroutineName = a subroutine with the interface as above, and 
\item description = a CHARACTER(LEN=128) character string that names the test
\end{itemize}

\item Setting the output location
	{\color{blue}\begin{verbatim}
         CALL tester % setOutputUnit(iUnit)
	\end{verbatim}}
\item Running tests
	{\color{blue}\begin{verbatim}
         CALL tester % performTests()
	\end{verbatim}}
\item Finalizing the test suite
	{\color{blue}\begin{verbatim}
         CALL tester % finalize()
	\end{verbatim}}

\end{itemize}

\subsection{Exceptions}
      An FTException object provides a way to pass generic
      information about an exceptional situation. Methods for
      dealing with exceptions are defined in the SharedExceptionManagerModule
      module.

      An FTException object wraps:
      \begin{itemize}
      \item A severity indicator
      \item A name for the exception
      \item An optional dictionary that contains whatever information is deemed necessary.

      It is expected that classes will define exceptions that use instances
      of the FTException Class.
      \end{itemize}
      
      Defined constants:
{\color{blue}\begin{verbatim}
         FT_ERROR_NONE    = 0
         FT_ERROR_WARNING = 1
         FT_ERROR_FATAL   = 2
\end{verbatim}}

     {\bf Usage:}
\begin{itemize}
\item Initialization
	{\color{blue}\begin{verbatim}
            e  %  initFTException(severity,exceptionName,infoDictionary)

            Plus the convenience initializers, which automatically 
            create a FTValueDictionary with a single key called "message":

            e % initWarningException(msg = "message")
            e % initFatalException(msg = "message")
	\end{verbatim}}
	
\item Setting components
   Create subroutines with the interface
{\color{blue}\begin{verbatim}
            e  %  setInfoDictionary(infoDictionary)
\end{verbatim}}

\end{itemize}
\end{document}  